\documentclass{law-pt-br}

\begin{document}

	\nomelei{CONSTITUIÇÃO DA REPÚBLICA FEDERATIVA DO BRASIL DE 1988}	

\begin{abstract}
	Nós, representantes do povo brasileiro, reunidos em Assembléia Nacional Constituinte para instituir um Estado Democrático, destinado a assegurar o exercício dos direitos sociais e individuais, a liberdade, a segurança, o bem-estar, o desenvolvimento, a igualdade e a justiça como valores supremos de uma sociedade fraterna, pluralista e sem preconceitos, fundada na harmonia social e comprometida, na ordem interna e internacional, com a solução pacífica das controvérsias, promulgamos, sob a proteção de Deus, a seguinte CONSTITUIÇÃO DA REPÚBLICA FEDERATIVA DO BRASIL.
\end{abstract}
	
	\titulo{Dos Princípios Fundamentais }
	
	
	\artigo{A República Federativa do Brasil, formada pela união indissolúvel dos Estados e Municípios e do Distrito Federal, constitui-se em Estado Democrático de Direito e tem como fundamentos:}
	
	\inciso{a soberania;}
	
	\inciso{a cidadania;}
	
	\inciso{a dignidade da pessoa humana;}
	
	\inciso{os valores sociais do trabalho e da livre iniciativa;}
	
	\inciso{o pluralismo político.}
    
   	\paragrafo{Todo o poder emana do povo, que o exerce por meio de representantes eleitos ou diretamente, nos termos desta Constituição.}
	
	\artigo{São Poderes da União, independentes e harmônicos entre si, o Legislativo, o Executivo e o Judiciário.}
	
	\artigo{Constituem objetivos fundamentais da República Federativa do Brasil:}
	
	\inciso{construir uma sociedade livre, justa e solidária;}
	
	\inciso{garantir o desenvolvimento nacional;}
	
	\inciso{erradicar a pobreza e a marginalização e reduzir as desigualdades sociais e regionais;}
	
	\inciso{promover o bem de todos, sem preconceitos de origem, raça, sexo, cor, idade e quaisquer outras formas de discriminação.}
    \section{outra seção}
    \artigo{}
\end{document}
